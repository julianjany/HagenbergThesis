%% Ein einfaches Template für einen Übungsbericht unter Verwendung des Hagenberg
%% Setups, basierend auf der LaTeX 'report' Standardklasse.
%%% äöüÄÖÜß  <-- keine deutschen Umlaute hier? UTF-faehigen Editor verwenden!

%%% Magic Comments zum Setzen der korrekten Parameter in kompatiblen IDEs
% !TeX encoding = utf8
% !TeX program = pdflatex
% !TeX spellcheck = de_DE
% !BIB program = biber

\documentclass[german,notitlepage,smartquotes]{hgbreport}
% Zulässige Optionen in [..]:
%    Hauptsprache: 'german' (default), 'english'
%    Option zur Umwandlung in typografische Anführungszeichen: 'smartquotes'
%    APA Zitierstil: 'apa'
%    Erzeuge keine separate Titelseite: 'notitlepage'
%%%-----------------------------------------------------------------------------

\RequirePackage[utf8]{inputenc} % bei Verw. von lualatex oder xelatex entfernen!

\renewcommand{\chapter}[1]{} % Deaktiviere den \chapter Befehl
\graphicspath{{images/}}     % Verzeichnis mit Bildern und Grafiken
\bibliography{references}    % Biblatex-Literaturdatei (references.bib)
\ExecuteBibliographyOptions{backref=false} % Keine Rückreferenzen bei Quellen

%%%-----------------------------------------------------------------------------
\setcounter{chapter}{1}	% <----- Auf die Übungsnummer setzen
%%%-----------------------------------------------------------------------------

\author{Julian Jany}                        % Name
\title{GP2 Generative Programmierung -- SS 2022\\ % Name der Übung
				Übungsabgabe \arabic{chapter}}
\date{\today}

%%%-----------------------------------------------------------------------------
\begin{document}
%%%-----------------------------------------------------------------------------
\maketitle
%%%-----------------------------------------------------------------------------

\begin{abstract}\noindent
\ldots
\end{abstract}

%%%-----------------------------------------------------------------------------

\section{Titel der ersten Aufgabe}

%%%-----------------------------------------------------------------------------

\section{Titel der zweiten Aufgabe}

%%%-----------------------------------------------------------------------------

\section*{Zusammenfassung und Anmerkungen}

%%%-----------------------------------------------------------------------------

% \section*{Quellen}

% \printbibliography[heading=noheader]

%%%-----------------------------------------------------------------------------
\end{document}
%%%-----------------------------------------------------------------------------